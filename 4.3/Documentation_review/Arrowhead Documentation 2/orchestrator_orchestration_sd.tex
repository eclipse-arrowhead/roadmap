\documentclass[a4paper]{arrowhead}

\usepackage[yyyymmdd]{datetime}
\usepackage{etoolbox}
\usepackage[utf8]{inputenc}
\usepackage{multirow}

\renewcommand{\dateseparator}{-}

%% Special references
\newcommand{\fref}[1]{{\textcolor{ArrowheadBlue}{\hyperref[sec:functions:#1]{#1}}}}
\newcommand{\mref}[1]{{\textcolor{ArrowheadPurple}{\hyperref[sec:model:#1]{#1}}}}
\newcommand{\pdef}[1]{{\textcolor{ArrowheadGrey}{#1\label{sec:model:primitives:#1}\label{sec:model:primitives:#1s}\label{sec:model:primitives:#1es}}}}
\newcommand{\pref}[1]{{\textcolor{ArrowheadGrey}{\hyperref[sec:model:primitives:#1]{#1}}}}

\newrobustcmd\fsubsection[3]{
  \addtocounter{subsection}{1}
  \addcontentsline{toc}{subsection}{\protect\numberline{\thesubsection}function \textcolor{ArrowheadBlue}{#1}}
  \renewcommand*{\do}[1]{\rref{##1},\ }
  \subsection*{
    \thesubsection\quad
    function
    \textcolor{ArrowheadBlue}{#1}
    (\notblank{#2}{\mref{#2}}{})
    \notblank{#3}{: \mref{#3}}{}
  }
  \label{sec:functions:#1}
}
\newrobustcmd\msubsection[2]{
  \addtocounter{subsection}{1}
  \addcontentsline{toc}{subsection}{\protect\numberline{\thesubsection}#1 \textcolor{ArrowheadPurple}{#2}}
  \subsection*{\thesubsection\quad#1 \textcolor{ArrowheadPurple}{#2}}
  \label{sec:model:#2} \label{sec:model:#2s} \label{sec:model:#2es}
}
%%

\begin{document}

%% Arrowhead Document Properties
\ArrowheadTitle{Orchestration}
\ArrowheadServiceID{orchestration}
\ArrowheadType{Service Description}
\ArrowheadTypeShort{SD}
\ArrowheadVersion{4.3.0}
\ArrowheadDate{\today}
\ArrowheadAuthor{Szvetlin Tanyi}
\ArrowheadStatus{RELEASE}
\ArrowheadContact{szvetlin@aitia.ai}
\ArrowheadFooter{\href{www.arrowhead.eu}{www.arrowhead.eu}}
\ArrowheadSetup
%%

%% Front Page
\begin{center}
  \vspace*{1cm}
  \huge{\arrowtitle}

  \vspace*{0.2cm}
  \LARGE{\arrowtype}
  \vspace*{1cm}

  \Large{Service ID: \textit{"\arrowid"}}
  \vspace*{\fill}

  % Front Page Image
  %\includegraphics{figures/TODO}

  \vspace*{1cm}
  \vspace*{\fill}

  % Front Page Abstract
  \begin{abstract}
    This document describes a variant of the Orchestration service.
  \end{abstract}

  \vspace*{1cm}

  \scriptsize
  \begin{tabularx}{\textwidth}{l X}
    \raisebox{-0.5\height}{\includegraphics[width=2cm]{figures/artemis_logo}} & {ARTEMIS Innovation Pilot Project: Arrowhead\newline
    THEME [SP1-JTI-ARTEMIS-2012-AIPP4 SP1-JTI-ARTEMIS-2012-AIPP6]\newline
    [Production and Energy System Automation Intelligent-Built environment and urban infrastructure for sustainable and friendly cities]}
  \end{tabularx}
  \vspace*{-0.2cm}
\end{center}
\newpage
%%

%% Table of Contents
\tableofcontents
\newpage
%%

\section{Overview}
\label{sec:overview}

This document describes the Orchestration Eclipse Arrowhead service, which provides Application Systems with orchestration information: where they need to connect to. The outcome of the Orchestration Service include rules that will tell the Application System what Service provider System(s) it should connect to and how (acting as a Service Consumer).

The rest of this document is organized as follows.
In Section \ref{sec:functions}, we describe the abstract message functions provided by the service.
In Section \ref{sec:model}, we end the document by presenting the data types used by the mentioned functions.

\newpage

\section{Service Interfaces}
\label{sec:functions}

This section lists the functions that must be exposed by the Public Key service in alphabetical order.
In particular, each subsection names an abstract interface, an input type and an output type, in that order.
The input type is named inside parentheses, while the output type is preceded by a colon.
Input and output types are only denoted when accepted or returned, respectively, by the interface in question.

All abstract data types named in this section are defined in Section \ref{sec:model}.

\fsubsection{Orchestration}{Orchestration}{}

This function enables orchestration for application systems to use.

\begin{figure}
  \centering
  \includegraphics[width=\textwidth,height=\textheight,keepaspectratio]{figures/post_orchestration_activity_uml.png}
  \caption{
    Information model as a UML activity diagram. Describes the process of Orchestration.
  }
  \label{fig:query_overview}
\end{figure}

\fsubsection{OrchestrationStoreManagement}{OrchestrationStoreManagement}{}

The function enables management of Store based orchestration rules.

\begin{figure}
  \centering
  \includegraphics[width=\textwidth,height=\textheight,keepaspectratio]{figures/post_store_orchestration_activity_uml.png}
  \caption{
    Information model as a UML activity diagram. Describes the process of Store Orchestration.
  }
  \label{fig:query_overview}
\end{figure}

\section{Information Model}
\label{sec:model}

Here, all data objects that can be part of Orchestration service calls are listed in alphabetic order.
Note that each subsection, which describes one type of object, begins with the \textit{struct} keyword, which is used to denote a collection of named fields, each with its own data type.
As a complement to the explicitly defined types in this section, there is also a list of implicit primitive types in Section \ref{sec:model:primitives}, which are used to represent things like hashes and identifiers.

\msubsection{struct}{ServiceRequestForm}

This structure is used to check whether the consumer system can use a service from a list of provider systems.

\begin{table}[ht!]
\begin{tabularx}{\textwidth}{| p{5cm} | p{5cm} | X |} \hline
\rowcolor{gray!33} Object Field & Value Type      & Description \\ \hline
"requesterSystem"                   & \pref{System}     & Requester System. \\ \hline
"requestedService"                  & \pref{Service}   & Requested Service. \\ \hline
"preferredProviders"                & \pref{Array}$<$\mref{PreferredProvider}$>$   & Array of Preferred providers \\ \hline
"orchestrationFlags"                & \pref{Object}     & Orchestration flags \\ \hline
"qosRequirements"                   & \pref{Object}     & QoS Requirements \\ \hline
"commands"                           & \pref{Object}     & Orchestration Commands \\ \hline

\end{tabularx}
\end{table}

\subsection{Primitives}
\label{sec:model:primitives}

Types and structures mentioned throughout this document that are assumed to be available to implementations of this service.
The concrete interpretations of each of these types and structures must be provided by any IDD document claiming to implement this service.

\begin{table}[ht!]
\begin{tabularx}{\textwidth}{| p{3cm} | X |} \hline
\rowcolor{gray!33} JSON Type & Description \\ \hline
\pdef{Value}                 & Any out of \pref{Object}, \pref{Array}, \pref{String}, \pref{Number}, \pref{Boolean} or \pref{Null}. \\ \hline
\pdef{Object}$<$A$>$         & An unordered collection of $[$\pref{String}: \pref{Value}$]$ pairs, where each \pref{Value} conforms to type A. \\ \hline
\pdef{Array}$<$A$>$          & An ordered collection of \pref{Value} elements, where each element conforms to type A. \\ \hline
\pdef{String}                & An arbitrary UTF-8 string. \\ \hline
\pdef{Number}                & Any IEEE 754 binary64 floating point number \cite{cowlishaw2019floating}, except for \textit{+Inf}, \textit{-Inf} and \textit{NaN}. \\ \hline
\pdef{Boolean}               & One out of \texttt{true} or \texttt{false}. \\ \hline
\pdef{Null}                  & Must be \texttt{null}. \\ \hline
\end{tabularx}
\end{table}

\newpage

\bibliographystyle{IEEEtran}
\bibliography{bibliography}

\newpage

\section{Revision History}
\subsection{Amendments}

\noindent\begin{tabularx}{\textwidth}{| p{1cm} | p{3cm} | p{2cm} | X | p{4cm} |} \hline
\rowcolor{gray!33} No. & Date & Version & Subject of Amendments & Author \\ \hline

1 & 2020-12-05 & 4.3.0 & & Tanyi Szvetlin \\ \hline

\end{tabularx}

\subsection{Quality Assurance}

\noindent\begin{tabularx}{\textwidth}{| p{1cm} | p{3cm} | p{2cm} | X |} \hline
\rowcolor{gray!33} No. & Date & Version & Approved by \\ \hline

1 & 2021-01-29 & 4.3.0 & Jerker Delsing\\ \hline

\end{tabularx}

\end{document}