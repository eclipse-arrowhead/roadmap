\documentclass[a4paper]{arrowhead}

\usepackage[yyyymmdd]{datetime}
\usepackage{etoolbox}
\usepackage[utf8]{inputenc}
\usepackage{multirow}

\renewcommand{\dateseparator}{-}

%% Special references
\newcommand{\fref}[1]{{\textcolor{ArrowheadBlue}{\hyperref[sec:functions:#1]{#1}}}}
\newcommand{\mref}[1]{{\textcolor{ArrowheadPurple}{\hyperref[sec:model:#1]{#1}}}}
\newcommand{\pdef}[1]{{\textcolor{ArrowheadGrey}{#1\label{sec:model:primitives:#1}\label{sec:model:primitives:#1s}\label{sec:model:primitives:#1es}}}}
\newcommand{\pref}[1]{{\textcolor{ArrowheadGrey}{\hyperref[sec:model:primitives:#1]{#1}}}}

\newrobustcmd\fsubsection[3]{
  \addtocounter{subsection}{1}
  \addcontentsline{toc}{subsection}{\protect\numberline{\thesubsection}function \textcolor{ArrowheadBlue}{#1}}
  \renewcommand*{\do}[1]{\rref{##1},\ }
  \subsection*{
    \thesubsection\quad
    operation
    \textcolor{ArrowheadBlue}{#1}
    (\notblank{#2}{\mref{#2}}{})
    \notblank{#3}{: \mref{#3}}{}
  }
  \label{sec:functions:#1}
}
\newrobustcmd\msubsection[2]{
  \addtocounter{subsection}{1}
  \addcontentsline{toc}{subsection}{\protect\numberline{\thesubsection}#1 \textcolor{ArrowheadPurple}{#2}}
  \subsection*{\thesubsection\quad#1 \textcolor{ArrowheadPurple}{#2}}
  \label{sec:model:#2} \label{sec:model:#2s} \label{sec:model:#2es}
}
%%

\begin{document}

%% Arrowhead Document Properties
\ArrowheadTitle{Eclispe Arrohead Naming Convention} % XXX = SystemName e.g. Service Registry HTTP/TLS/JSON}
\ArrowheadType{}
\ArrowheadTypeShort{SoSD}
\ArrowheadVersion{5.0.0} % Arrowhead version X.Y.Z, e..g. 4.4.1
\ArrowheadDate{\today}
\ArrowheadAuthor{Jerker Delsing} % Corresponding author e.g. Jerker Delsing
\ArrowheadStatus{Prototype} % e..g. RELEASE, RELEASE CONDIDATE, PROTOTYPE
\ArrowheadContact{jerker.delsing@ltu.se} % Email of corresponding author
\ArrowheadFooter{\href{www.arrowhead.eu}{www.arrowhead.eu}}
\ArrowheadSetup
%%

%% Front Page
\begin{center}
  \vspace*{1cm}
  \huge{\arrowtitle}

  \vspace*{0.2cm}
  \LARGE{\arrowtype}
  \vspace*{1cm}

  %\Large{Service ID: \textit{"\arrowid"}}
  \vspace*{\fill}

  % Front Page Image
  %\includegraphics{figures/TODO}

  \vspace*{1cm}
  \vspace*{\fill}

  % Front Page Abstract
  \begin{abstract}
    Proposla for naming convention of microsystems, microservices and
    associated attributes and metadata. This is intended as an
    appendix to the Eclipse Arrowhead GSoSD document.  
  \end{abstract}

  \vspace*{1cm}

%   \scriptsize
%   \begin{tabularx}{\textwidth}{l X}
%     \raisebox{-0.5\height}{\includegraphics[width=2cm]{figures/artemis_logo}} & {ARTEMIS Innovation Pilot Project: Arrowhead\newline
%     THEME [SP1-JTI-ARTEMIS-2012-AIPP4 SP1-JTI-ARTEMIS-2012-AIPP6]\newline
%     [Production and Energy System Automation Intelligent-Built environment and urban infrastructure for sustainable and friendly cities]}
%   \end{tabularx}
%   \vspace*{-0.2cm}
 \end{center}

\newpage
%%

%% Table of Contents
\tableofcontents
\newpage
%%

\section{Overview}
\label{sec:overview}
  Proposla for naming convention of microsystems, microservices and
    associated attributes and metadata. This is intended as an
    appendix to the Eclipse Arrowhead GSoSD document.

\subsection{Why naming convention is necessary}
System, service, operation, device and interface names are used in
programing languages, URLs, DNS entries, file systems. A consistent
approach to naming in the community will simplify usages of different programing
languages and support human and machine understanding of the intended
meaning. Different key entities in the Arrowhead architecture should
easily be distinguished throug its naming style.  
It will further support integration and interoperability with
major internet and modeling technologies abd standards, like
e.g. Semantic Web, Ontologies, Knowledge Graphs, SysML. It will further support interoperability and
integration with major industrial standards like e.g. ISO 15926, ISO
10303, S5000, IEC 81346. 
    
The rest of this document is organized as follows.
In Section \ref{sec:prior_art}, we reference major prior art on
microsystem and microservice naming within the Eclispe Arrowhead project. 

In Section \ref{sec:principles}, we detail the underlaying thinking
and principles for the
naming convention.

In Section \ref{sec:examples}, we provide a set of example.


\newpage

\section{Significant Prior Art}
\label{sec:prior_art}

A previous proposal by Paniagua et.al \cite{Paniagua_2019} has not
  gained attention. Thus we here proposa a significantly simpler
  naming convention approach for Eclipse Arrowhead mincrosystems,
  microservice and associated metadata and attributes.

\section{Foundational naming principles}
\label{sec:principles}
The ambition with this naming conventions is to provide names being:
\begin{itemize}

\item Names shall only be composed of ASCII characters.

\item Names shall reflect the intended functionality and usage in an
  SOA architecture
  
\item Different style per architecture entity
  \begin{itemize}
  \item System name: PascalCase
    \begin{itemize}
    \item Example: ServiceRegistry
    \end{itemize}

  \item Service name: camelCase
  \begin{itemize}
    \item Example: serviceDiscovery
    \end{itemize}

  \item Service operation name: kebab-case
  \begin{itemize}
    \item Example: get-entries
    \end{itemize}

  \item Interface name: snake\_case
    \begin{itemize}
    \item Example: generic\_http
    \end{itemize}

  \item Device name: UPPER\_SNAKE\_CASE
  \begin{itemize}
    \item Example: MY\_DEVICE
    \end{itemize}
    
  \item Attributes like e.g. keys in payloads, metadata, properties: camelCase
   \begin{itemize}
    \item Example: subUrl
    \end{itemize}
 \end{itemize}

\item Composite identifiers like service instance identifiers: \\
  SystemName $<$delimiter$>$serviceName$<$delimiter$>$version \\
  Cloud identifiers: CloudName$<$delimiter$>$Organization \\
  delimiter: ``$|$”  is proposed
   \begin{itemize}
    \item Examples: \\
      ServiceRegistry$|$serviceDiscovery$|$1.0.0 \\
      TestCloud$|$AitiaInc)
     \end{itemize}
 
  

 
\item Naming of instances of microsystems microservices, metadata and
  attrtibutes shall follow  the naming convention of the
  applied standard e.g. ISO 15296, ISO 10303, S5000
  
\item The choice of industry standards to be applied should preferable
  be possible to connected to the Industrial Data Ontology (IDO), ISO 23726-3


\end{itemize}


\section{Naming example}
\label{sec:examples}

Please find below a set of examples for the most important naming
siutaitons in the Eclipse Arrowhead architecture
\begin{itemize}
\item Microsystems name - PacalCase
  \begin{itemize}
  \item ServiceRegistry
  \item DynamicServiceOrchestration
  \item SimpleServiceOrchestration
  \item FlexibleServiceOrchestration
  \item ComputeOrchestrationSystem
  \item DeploymentOrchestrationSystem
  \item ConsumerAuhtorizationSystem
  \item Authentication
  \end{itemize}
  
\item Microservices name - camelCaseal
  \begin{itemize}
  \item serviceDiscovery
  \item serviceOrchestration
  \item computeOrchestration
  \item simpleOrchestrationStoreManagement
  \item flexibleOrchestrationStoreManagement
  \item deploymentOrchestration
  \item consumerAuhtorization
  \end{itemize}

 \item Metadata and attribute naming: camelCase (which in combination to the
  related Microsystem or Micsroservice shall be meaningfull)
  \begin{itemize}
  \item Metadata/attribute: timeStamp (of what should be possible to infere from
    the naming of the microsystem or microservice instance to which the
    metadata/attribute is connected)
  \item Metadata/attribute: softWareVersion (version reference of the
    deployed software)
  \item Metadata/attribute: compiler (which comiler and verson was used)
  \item Metadata/attribute: compilerSwitches (used compiler switches
    and value) 
  \end{itemize}

\end{itemize} 

\section{How to align with other standards in use}

\begin{itemize}

\item URL Reserved characters: ! * ' ( ) ; : @ \& = + \$ , / ? \# [ ] 
  
  
\item Not to use these as separator in composite identifiers.
 \item Certificate authentication method: X.509 certificate common name allows only: uppercase letters, lowercase letters, digits, hyphen (but not at the start or end) and dot (as a separator between domain levels)
   \begin{itemize}
   \item To use kebab-case in certificates and transform in code
     level (service-registry.test-cloud.aitia-inc.arrowhead.eu =
     ServiceRegistry.TestCloud.AitiaInc.arrowhead.eu) 
   \item Not to use dot as separator in composite identifiers.\\
     CN represents a fully qualified domain name, which can be maximum
     253 character and max 63 char per label.  (subdomain.example.com)
     
   \item To apply max 63 char rule to the names
   \item Semantic versioning: $<$major$>$.$<$minor$>$.$<$patch$>$
   \item Not to use dot as separator in composite identifiers.
   \item RDF (Resource Description Framework) for Knowledge Graphs: Reserved characters: $<$, $>$, \&, “, \\
     Not to use these as separator in composite identifiers.
   \end{itemize}
 \end{itemize}
 
\bibliographystyle{IEEEtran}
\bibliography{bibliography}

\newpage

\section{Revision History}
%\subsection{Amendments}


\noindent\begin{tabularx}{\textwidth}{| p{1cm} | p{3cm} | p{2cm} | X | p{4cm} |} \hline
\rowcolor{gray!33} No. & Date & Version & Subject of Amendments & Author \\ \hline

1 & 2025-04-02 & \arrowversion & & Jerker Delsing \\ \hline
2 & 2025-04-09 & \arrowversion & & Jerker Delsing \\ \hline
3 & & & & \\ \hline

\end{tabularx}

\subsection{Quality Assurance}

\noindent\begin{tabularx}{\textwidth}{| p{1cm} | p{3cm} | p{2cm} | X |} \hline
\rowcolor{gray!33} No. & Date & Version & Approved by \\ \hline

1 & 2025-04-02 & \arrowversion  &  Pal Varga\\ \hline

\end{tabularx}

\end{document}