\documentclass[a4paper]{arrowhead}

\usepackage[yyyymmdd]{datetime}
\usepackage{etoolbox}
\usepackage[utf8]{inputenc}
\usepackage{multirow}

\renewcommand{\dateseparator}{-}

%% Special references
\newcommand{\fref}[1]{{\textcolor{ArrowheadBlue}{\hyperref[sec:functions:#1]{#1}}}}
\newcommand{\mref}[1]{{\textcolor{ArrowheadPurple}{\hyperref[sec:model:#1]{#1}}}}
\newcommand{\pdef}[1]{{\textcolor{ArrowheadGrey}{#1\label{sec:model:primitives:#1}\label{sec:model:primitives:#1s}\label{sec:model:primitives:#1es}}}}
\newcommand{\pref}[1]{{\textcolor{ArrowheadGrey}{\hyperref[sec:model:primitives:#1]{#1}}}}

\newrobustcmd\fsubsection[3]{
  \addtocounter{subsection}{1}
  \addcontentsline{toc}{subsection}{\protect\numberline{\thesubsection}function \textcolor{ArrowheadBlue}{#1}}
  \renewcommand*{\do}[1]{\rref{##1},\ }
  \subsection*{
    \thesubsection\quad
    interface operation
    \textcolor{ArrowheadBlue}{#1}
    (\notblank{#2}{\mref{#2}}{})
    \notblank{#3}{: \mref{#3}}{}
  }
  \label{sec:functions:#1}
}
\newrobustcmd\msubsection[2]{
  \addtocounter{subsection}{1}
  \addcontentsline{toc}{subsection}{\protect\numberline{\thesubsection}#1 \textcolor{ArrowheadPurple}{#2}}
  \subsection*{\thesubsection\quad#1 \textcolor{ArrowheadPurple}{#2}}
  \label{sec:model:#2} \label{sec:model:#2s} \label{sec:model:#2es}
}
%%

\begin{document}

%% Arrowhead Document Properties
\ArrowheadTitle{SysMonitor} % XXX = SystemName e.g. Service Registry HTTP/TLS/JSON}
\ArrowheadType{System Description}
\ArrowheadTypeShort{SysD}
\ArrowheadVersion{5.0.0} % Arrowhead version X.Y.Z, e..g. 4.4.1
\ArrowheadDate{\today}
\ArrowheadAuthor{Jerker Delsing} % Corresponding author e.g. Jerker Delsing
\ArrowheadStatus{PROTOTYPE} % e..g. RELEASE, RELEASE CONDIDATE, PROTOTYPE
\ArrowheadContact{jerker.delsing@ltu.se} % Email of corresponding author
\ArrowheadFooter{\href{www.arrowhead.eu}{www.arrowhead.eu}}
\ArrowheadSetup
%%

%% Front Page
\begin{center}
  \vspace*{1cm}
  \huge{\arrowtitle}

  \vspace*{0.2cm}
  \LARGE{\arrowtype}
  \vspace*{1cm}

  %\Large{Service ID: \textit{"\arrowid"}}
  \vspace*{\fill}

  % Front Page Image
  %\includegraphics{figures/TODO}

  \vspace*{1cm}
  \vspace*{\fill}

  % Front Page Abstract
  \begin{abstract}
    The SysMonitor microsystem is a support system in the Eclipse
    Arrowhead suite of SOA/Microservie infrastructure. The objective
    of the SysMonitor system is to consume all produced Monitor
    microservices in a local cloud and making monitored data avialable
    for furhter processing.
  \end{abstract}

  \vspace*{1cm}

%   \scriptsize
%   \begin{tabularx}{\textwidth}{l X}
%     \raisebox{-0.5\height}{\includegraphics[width=2cm]{figures/artemis_logo}} & {ARTEMIS Innovation Pilot Project: Arrowhead\newline
%     THEME [SP1-JTI-ARTEMIS-2012-AIPP4 SP1-JTI-ARTEMIS-2012-AIPP6]\newline
%     [Production and Energy System Automation Intelligent-Built environment and urban infrastructure for sustainable and friendly cities]}
%   \end{tabularx}
%   \vspace*{-0.2cm}
 \end{center}

\newpage
%%

%% Table of Contents
\tableofcontents
\newpage
%%

\section{Overview}
\label{sec:overview}
  The SysMonitor microsystem is a support system in the Eclipse
    Arrowhead suite of SOA/Microservie infrastructure. The objective
    of the SysMonitor system is to consume all produced Monitor
    microservices in a local cloud and making monitored data avialable
    for furhter processing addressing SoS issues like e.g. resilience,
    failover, maintenance. 
    
The rest of this document is organized as follows.
In Section \ref{sec:prior_art}, we reference major prior art capabilitites
of the system.
In Section \ref{sec:use}, we the intended usage of the system.
In Section \ref{sec:properties}, we describe fundmental properties
provided by the system.
In Section \ref{sec:delimitations}, we describe de-limitations of capabilitites
ofn the system.
In Section \ref{sec:services}, we describe the abstract service
functions consumed or produced by the system.
In Section \ref{sec:security}, we describe the security capabilitites
of the system.

%\newpage

\subsection{Significant Prior Art}
\label{sec:prior_art}

In the edge SOA/microservice domain very little is avilable from the
major open source projects. In legacy and monolitic code monitoring is
mainstream but most often propritary knowledge and code. 

\subsection{How This System Is Meant to Be Used}
\label{sec:use}

The intended usage 
    of the SysMonitor system is to consume all produced Monitor
    microservices in a local cloud and making monitored data avialable
    for furhter processing addressing SoS issues like e.g. resilience,
    failover, maintenance. The consumption of Monitor services shall
    be on a publish subscribe basis (MQTT).  
    


\subsection{System functionalities and properties}
\label{sec:properties}

Consumption of all produced Monitor service registered in the
ServiceRegistry of a local cloud. Such
services will provide a few mandatory interfaces and evnetually
additional interfaces of importance to the hosting microsystem. It is
required that non-mandatory interface shall provide necessary metadata
such that the SysMonitor system can generate a complete and working publish subscribe
consumer. The subscription to indivudual Monitor service shall be
possible to condtion as allowed by the prodicing microsystem.




\subsubsection {Functional properties of the system}

Consumption on a publish subscribe basis of produced Monitor services.

Capability to generate Monitor service consumers for those produiced
Monitor service having non-mandatory interface with necessary meta
data stored int the ServiceRegistry. Such generated code should be
possible to V\&V in run-time.  


\subsubsection {Configuration of system properties}

Subscription condition to consumed Monitor service shall be possible
to configure using a Monitor\_configuration service.

\subsubsection {Data stored by the system}

Monitored data will be pushed to the DataManager making data available
for further processing.  

\subsubsection {Non functional properties}
\begin{itemize}
  \item security, AA security based on x.509 certificates is supported. 
  \item safety, ...
  \item energy consumption,  ...
  \item latency,  ...
  \item Power saving properties,  ... 
\end{itemize}


\subsubsection {Stateful or stateless}
\begin{itemize} 
\item states preserved, functional and non-functional
\end{itemize}  


\subsection{Important Delimitations}
\label{sec:delimitations}

The SysMonitor system is not expected to take actions on the monitor data
being collected. Data will be pushed to DataManager for further
actions by other microsystems within the local cloud or external local clouds.


%\newpage

\section{Services}
\label{sec:services}

This section describes consumed and produced service.
In particular, each subsection names a prodiuced or consumed service
indicating the different capabilities and associated interfaces of the
service. Reference to the appropriate SD document shall be made.

\subsection{Produced service}

\begin{itemize}
\item Monitor\_management
\end{itemize}

\subsection{Consumed services}

The core services:
\begin{itemize}
\item ServiceDsicovery
\item Authorisation
\item Authentication
\item Orchestration
\end{itemize}

and all Monitor services registered in the ServiceRegistry.


%\newpage

\section{Security}
\label{sec:security}


Overview of security level chosen for the system

The following is covered 
\begin{itemize}
\item  The system can be started in un-secure and/or
Arrowhead secure mode.
\item The system can only handling Arrowhead compliant X.509 certificates.
\end{itemize}

\subsection {Security model}
The following points should be described:
\begin{itemize}
\item protocol supported: HTTP, MQTT 
\item data protection supported: TLS 
\item system authentication capability supported: Arrowhead X.509 certificate
\item produced service authorisation checking: via token from
  Orchestration system directly via the Authorisation system 

\end{itemize}

For Arrowhead certificate profile
see github.com/eclipse-arrowhead/documentation




\bibliographystyle{IEEEtran}
\bibliography{bibliography}

\newpage

\section{Revision History}
\subsection{Amendments}

Revision history and Quality assurance. 

\noindent\begin{tabularx}{\textwidth}{| p{1cm} | p{3cm} | p{2cm} | X | p{4cm} |} \hline
\rowcolor{gray!33} No. & Date & Version & Subject of Amendments & Author \\ \hline

1 & 2024-05-26 & \arrowversion & & Jerker Delsing \\ \hline
2 &  & \arrowversion & & \\ \hline
3 & & \arrowversion & & \\ \hline
\end{tabularx}

\subsection{Quality Assurance}

\noindent\begin{tabularx}{\textwidth}{| p{1cm} | p{3cm} | p{2cm} | X |} \hline
\rowcolor{gray!33} No. & Date & Version & Approved by \\ \hline

1 & 2022-01-10 & \arrowversion  &  \\ \hline

\end{tabularx}

\end{document}