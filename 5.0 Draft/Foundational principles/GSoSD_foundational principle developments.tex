\documentclass[a4paper]{arrowhead}

\usepackage[yyyymmdd]{datetime}
\usepackage{etoolbox}
\usepackage[utf8]{inputenc}
\usepackage{multirow}
\usepackage{url}

\renewcommand{\dateseparator}{-}

%% Special references
\newcommand{\fref}[1]{{\textcolor{ArrowheadBlue}{\hyperref[sec:functions:#1]{#1}}}}
\newcommand{\mref}[1]{{\textcolor{ArrowheadPurple}{\hyperref[sec:model:#1]{#1}}}}
\newcommand{\pdef}[1]{{\textcolor{ArrowheadGrey}{#1\label{sec:model:primitives:#1}\label{sec:model:primitives:#1s}\label{sec:model:primitives:#1es}}}}
\newcommand{\pref}[1]{{\textcolor{ArrowheadGrey}{\hyperref[sec:model:primitives:#1]{#1}}}}

\newrobustcmd\fsubsection[3]{
  \addtocounter{subsection}{1}
  \addcontentsline{toc}{subsection}{\protect\numberline{\thesubsection}function \textcolor{ArrowheadBlue}{#1}}
  \renewcommand*{\do}[1]{\rref{##1},\ }
  \subsection*{
    \thesubsection\quad
    operation
    \textcolor{ArrowheadBlue}{#1}
    (\notblank{#2}{\mref{#2}}{})
    \notblank{#3}{: \mref{#3}}{}
  }
  \label{sec:functions:#1}
}
\newrobustcmd\msubsection[2]{
  \addtocounter{subsection}{1}
  \addcontentsline{toc}{subsection}{\protect\numberline{\thesubsection}#1 \textcolor{ArrowheadPurple}{#2}}
  \subsection*{\thesubsection\quad#1 \textcolor{ArrowheadPurple}{#2}}
  \label{sec:model:#2} \label{sec:model:#2s} \label{sec:model:#2es}
}
%%


\begin{document}

%% Arrowhead Document Properties
\ArrowheadTitle{Fundational principles} % XXX = SystemName e.g. Service Registry HTTP/TLS/JSON}
\ArrowheadType{Generic System of Systems Description - }
\ArrowheadTypeShort{GSoSD-F}
\ArrowheadVersion{4.6.1} % Arrowhead version X.Y.Z, e..g. 4.4.1
\ArrowheadDate{\today}
\ArrowheadAuthor{Jerker Delsing} % Corresponding author e.g. Jerker Delsing
\ArrowheadStatus{DRAFT} % e..g. RELEASE, RELEASE CONDIDATE, PROTOTYPE
\ArrowheadContact{jerker.delsing@ltu.se} % Email of corresponding author
\ArrowheadFooter{\href{www.arrowhead.eu}{www.arrowhead.eu}}
\ArrowheadSetup
%%

%% Front Page
\begin{center}
  \vspace*{1cm}
  \huge{\arrowtitle}

  \vspace*{0.2cm}
  \LARGE{\arrowtype}
  \vspace*{1cm}

  %\Large{Service ID: \textit{"\arrowid"}}
  \vspace*{\fill}

  % Front Page Image
  %\includegraphics{figures/TODO}

  \vspace*{1cm}
  \vspace*{\fill}

  % Front Page Abstract
  \begin{abstract}
    This is the template for System Description (SysD document)
    according to the Eclipse Arrowehad documentation structure. 
  \end{abstract}

%  \Arrowhead*{1cm}

%   \scriptsize
%   \begin{tabularx}{\textwidth}{l X}
%     \raisebox{-0.5\height}{\includegraphics[width=2cm]{figures/artemis_logo}} & {ARTEMIS Innovation Pilot Project: Arrowhead\newline
%     THEME [SP1-JTI-ARTEMIS-2012-AIPP4 SP1-JTI-ARTEMIS-2012-AIPP6]\newline
%     [Production and Energy System Automation Intelligent-Built environment and urban infrastructure for sustainable and friendly cities]}
%   \end{tabularx}
%   \vspace*{-0.2cm}
 \end{center}

\newpage
%%

%% Table of Contents
\tableofcontents
\newpage
%%

\section{Overview}
\label{sec:overview}
This document describes the fundational principels for Eclispe
Arrowhead. 

The rest of this document is organized as follows.
In Section \ref{sec:prior_art}, we reference major prior art capabilitites
of the system.
In Section \ref{sec:use}, we the intended usage of the system.
In Section \ref{sec:properties}, we describe fundmental properties
provided by the system.
In Section \ref{sec:delimitations}, we describe de-limitations of capabilitites
ofn the system.
In Section \ref{sec:services}, we describe the abstract service
functions consumed or produced by the system.
In Section \ref{sec:security}, we describe the security capabilitites
of the system.

\newpage

\subsection{Significant Prior Art}
\label{sec:prior_art}

Eclipse Arrowhead has its roots in service oriented arcitecture, SOA,
and its use for primarily far edge, edge and fog automation and
digitalsiation with interoperability to the cloud level.

A set of EU projects have built the foundation for whats now Eclipse
Arrowhead, current version 4.6.1 with the specifications for v5.0 in
the works. These projects are:
\begin{itemize}
\item Socrades
\item IMC-AESOP
\item Arrowhead
\item Productive4.0
\item Arrowhead Tools
\item AIMS5.0
\item Arrowhead fPVN
\end{itemize}

A couple of basic architectture ideas has been around since the prior
art projects:
\begin{itemize}
\item Socrades:
  \begin{itemize}
  \item Hard real time control using internet protocols
  \end{itemize}

\item IMC-AESOP:
  \begin{itemize}
  \item Objective to be capable of implementing real world SCADA and
    DCS systems.
  \item The local cloud concept was born. Local clouds are self
    contined for its intended operation enabling local security,
    protection and if equiped with TDMA network MAC real time
    properties can be achived.
  \item System are self contained for its intended operation,
    e.g. owning and responsible for its own data storage and compitational resources.

  \item Arrowhead:
    \begin{itemize}
    \item Objective to be interoperability to legacy and internet protocols and being
      Open Source.
    \item Basic SOA foundation established, Look-up, Late binding and
      Lossely coupled.
    \item Mandatory core systems defined: ServiceRegistry,
      Orcehstration, Authorisation
    \item Interoperability enabled through translation dynamically
      instatiated when needed.
    \item v3.3 released as open source      
    \end{itemize}
  
    \item Productive4.0:
      \begin{itemize}
      \item Arrowhead Framework becomes Eclipse Arrowhead architecture
        and implementation platform
      \item Extending the implementation platform - teh Arrowhead
        technology stack is defined
      \item v4.5 released
      \end{itemize}

      
    \item Arowheead Tools
      \begin{itemize}
      \item Objective to reduce engineereing cost with 20-50\%
      \item Extending the Eclipse Arrowehad technology stack
      \item v4.6 released
      \item Achived 30-95\%engineering cost and time reduction in 28
        industrial use cases along the extended IEC 81346 engineering process. 
      \end{itemize}
    \end{itemize}
\end{itemize}


 The current comprehensive high level architecture description of
 Eclipse Arrowhead architecture is the book ``IoT Automation -
 Arrowhead framework''  \cite{Delsing2017a}. The currently released
 core systems and associated documentations
 are avialable at \url{www.github.com/eclipsearrowhead}.

 \begin{figure}[ht!]
   \centering
   \includegraphics[width=0.9\linewidth]{figures/arrowhead_technology_stack}
   \caption{The Eclipse Arrowhead technology stack an dassoicated
     microsystems, released, relase candidates and protoypes.}
   \label{fig:technology_stack}
 \end{figure}

\subsection{Eclispe Arrowhead architecture philophosy}
\label{sec:use}

The architecture philiposy is based on the following key technology
decission and objectives:
\begin{itemize}
\item Key technology decisions
  \begin{itemize}
  \item A fully distributed microservice SOA approach shall be used
  \item Support for design and run time engineering

  \item A set of microsystems, the technology stack cf. Figure
    \ref{fig:technology_stack}, shall be provided enabling the
    implementation of automation and digitalisation solutions
    
    \begin{itemize}
    \item Three core microsystems considered as primary and almost
      mandatory, ServiceRegistry, Orcehstration, Authorisation,
      enabling Look-up, Late binding and Lossely coupling.
    \item A set of support microsystem will be defined and implemented
      covering the technology stack cf. Figure enabling implementation
      automation architectures like ISA-95 and RAMI4.0. 
      \ref{fig:technology_stack}.
      \end{itemize}

    \item Basic microsystem properties: A micosystem can be stateless
      or statefull. If statefull the microsystem is responsible for
      its own data stoarage perferable using a database and a well
      established/standardised datamodel.

    \item The local cloud concept shall be used providing segmentation
      and protection of functionall properties enabling differentieade
      security, safety, and real time properties within a solution
      architecture with managed access into a segment and between segements.

    \item Network technology agnostic, allowing for different network
      properties inside different local clouds.

    
    \item Support for multipe strategy direction. A strategy
      directions may be: Security, LifeCycle, Maintenance, Business,
      Audit, Monitoring, BusinessAdminstration, BusinessModels. This
      will also require ways of addressing interdependecies between
      the various stratgy directions.
      
    \item Interoperability support at Service level shall be provided
      regrding: SOA, IP and legacy protocols, encodings, compressions,
      security, data models, trough translators or dedicated
      adaptors. For data model interopeability between major standards
      like e.g. ISO10303, ISO 15926, IEC 81346 are prioritised.
      
    \item Security shall be supported at service exchange level with
      authentication, authorisation and audit. Security at finer
      granularity that service level is being addressed.
      
    \item Secure on-boarding: On-boarding based on authentication of
      devices, microsystems and microservices shall be supproted.
      
 

    \end{itemize}

  
    \item A documentation structure has been defined, cf. Figure
      \ref{fig:documentation_structure}.  
    \item 
    \end{itemize}
  
\begin{figure}[ht!]
   \centering
   \includegraphics[width=0.9\linewidth]{figures/documentation_structure}
   \caption{The Eclipse Arrowhead documentation structure.}
   \label{fig:technology_stack}
 \end{figure}

\subsection{Ecllipse Arrowhead core systems}
\label{sec:core_systems}

\color{red}
Narrative describe system functionalities and properties (no
implmentation details) like e.g.:

\subsubsection {Functional properties of the system}

\subsubsection {Configuration of system properties}


\subsubsection {Data stored by the system}
Brief overview of data stored to achive the functionality of the system. 

\subsubsection {Non functional properties}
\begin{itemize}
  \item security, 
  \item safety, 
  \item energy consumption,
  \item latency
  \item Power saving properties, 
\end{itemize}


\subsubsection {Stateful or stateless}
\begin{itemize} 
\item states preserved, functional and non-functional
\end{itemize}  
\color{black}  


\subsection{Important Delimitations}
\label{sec:delimitations}

\color{red}
Provide delimitations of the provided system. Describe what the system
solve and what i does not solve.
\color{black}  



\newpage

\section{Services}
\label{sec:services}

\color{red}
This section describes consumed and produced service.
In particular, each subsection names a prodiuced or consumed service
indicating the different capabilities and associated interfaces of the
service. Reference to the appropriate SD document shall be made.

\subsection{Produced service}
with references to SD and IDD documents

\subsection{Consumed services}
with references to SD and IDD documents

\color{black}




\newpage

\section{Security}
\label{sec:security}


\color{red}
Overview of security leel chosen for the system

The follwoing bullets should be covered 
\begin{itemize}
\item  If the system can be started in un-secure and/or
Arrowhead secure mode.
\item Handling of Arrowhead compliant and
non-compliant X.509 certificates.
\end{itemize}

\subsection {Security model}
The following points should be described:
\begin{itemize}
\item protocol supported 
\item data protection supported 
\item system authentication capability supported
\item produced service authorisation checking, 
\item etc.
\end{itemize}

For Arrowhead certificate profile
see github.com/eclipse-arrowhead/documentation
\color{black}




\bibliographystyle{IEEEtran}
\bibliography{arrowhead-bib}


\newpage

\section{Revision History}
\subsection{Amendments}

Revision history and Quality assurance as per examples below\color{black}

\noindent\begin{tabularx}{\textwidth}{| p{1cm} | p{3cm} | p{2cm} | X | p{4cm} |} \hline
\rowcolor{gray!33} No. & Date & Version & Subject of Amendments & Author \\ \hline

1 & 2023-05-08 & \arrowversion & & Jerker Delsing \\ \hline
2 & & & & \\ \hline
3 & & & & \\ \hline
\end{tabularx}

\subsection{Quality Assurance}

\noindent\begin{tabularx}{\textwidth}{| p{1cm} | p{3cm} | p{2cm} | X |} \hline
\rowcolor{gray!33} No. & Date & Version & Approved by \\ \hline

1 & 2022-01-10 & \arrowversion  &  \\ \hline

\end{tabularx}

\end{document}
%  LocalWords:  Arrowehad
