\documentclass[a4paper]{arrowhead}

\usepackage[yyyymmdd]{datetime}
\usepackage{etoolbox}
\usepackage[utf8]{inputenc}
\usepackage{multirow}

\renewcommand{\dateseparator}{-}

\newcommand{\fparam}[1]{\textit{\textcolor{ArrowheadBlue}{#1}}}

%% Special references
\newcommand{\fref}[1]{{\textcolor{ArrowheadBlue}{\hyperref[sec:functions:#1]{#1}}}}
\newcommand{\mref}[1]{{\textcolor{ArrowheadPurple}{\hyperref[sec:model:#1]{#1}}}}
\newcommand{\pdef}[1]{{\textcolor{ArrowheadGrey}{#1 \label{sec:model:primitives:#1} \label{sec:model:primitives:#1s}}}}
\newcommand{\pref}[1]{{\textcolor{ArrowheadGrey}{\hyperref[sec:model:primitives:#1]{#1}}}}

\newrobustcmd\fsubsection[5]{
  \addtocounter{subsection}{1}
  \addcontentsline{toc}{subsection}{\protect\numberline{\thesubsection}function \textcolor{ArrowheadBlue}{#1}}
  \renewcommand*{\do}[1]{\rref{##1},\ }
  \subsection*{
    \thesubsection\quad
    #2 \textcolor{ArrowheadPurple}{#3} \\
    \small
    \hspace*{0.075\textwidth}\begin{minipage}{0.1\textwidth}
      \vspace*{1mm}
      Interface: \\
      \notblank{#4}{Input: \\}{}
      \notblank{#5}{Output: \\}{}
    \end{minipage}
    \begin{minipage}{0.825\textwidth}
      \vspace*{1mm}
      \textcolor{ArrowheadBlue}{#1} \\
      \notblank{#4}{\mref{#4} \\}{}
      \notblank{#5}{\mref{#5} \\}{}
    \end{minipage}
  }
  \label{sec:functions:#1}
}
\newrobustcmd\msubsection[2]{
  \addtocounter{subsection}{1}
  \addcontentsline{toc}{subsection}{\protect\numberline{\thesubsection}#1 \textcolor{ArrowheadPurple}{#2}}
  \subsection*{\thesubsection\quad#1 \textcolor{ArrowheadPurple}{#2}}
  \label{sec:model:#2} \label{sec:model:#2s}
}
%%

\begin{document}

%% Arrowhead Document Properties
\ArrowheadTitle{Event Unscribe HTTP/TLS/JSON}
\ArrowheadServiceID{event-unsubscribe}
\ArrowheadType{Interface Design Description}
\ArrowheadTypeShort{IDD}
\ArrowheadVersion{4.3.0}
\ArrowheadDate{\today}
\ArrowheadAuthor{Szvetlin Tanyi}
\ArrowheadStatus{RELEASE}
\ArrowheadContact{szvetlin@aitia.ai}
\ArrowheadFooter{\href{www.arrowhead.eu}{www.arrowhead.eu}}
\ArrowheadSetup
%%

%% Front Page
\begin{center}
  \vspace*{1cm}
  \huge{\arrowtitle}

  \vspace*{0.2cm}
  \LARGE{\arrowtype}
  \vspace*{1cm}

  \Large{Service ID: \textit{"\arrowid"}}
  \vspace*{\fill}

  % Front Page Image
  %\includegraphics{figures/TODO}

  \vspace*{1cm}
  \vspace*{\fill}

  % Front Page Abstract
  \begin{abstract}
    This document describes a HTTP/TLS/JSON variant of the Event Unsubscribe service.
  \end{abstract}

  \vspace*{1cm}

  \scriptsize
  \begin{tabularx}{\textwidth}{l X}
    \raisebox{-0.5\height}{\includegraphics[width=2cm]{figures/artemis_logo}} & {ARTEMIS Innovation Pilot Project: Arrowhead\newline
    THEME [SP1-JTI-ARTEMIS-2012-AIPP4 SP1-JTI-ARTEMIS-2012-AIPP6]\newline
    [Production and Energy System Automation Intelligent-Built environment and urban infrastructure for sustainable and friendly cities]}
  \end{tabularx}
  \vspace*{-0.2cm}
\end{center}
\newpage
%%

%% Table of Contents
\tableofcontents
\newpage
%%

\section{Overview}
\label{sec:overview}

This document describes the HTTP/TLS/JSON variant of the Event Unsubscribe Eclipse Arrowhead service, which enables unsubscribing from events.
Examples of this interaction is a consumer system would not like to receive a specific event anymore.

This document exists as a complement to the \textit{Event Unsubscribe -- Service Description} document.
For further details about how this service is meant to be used, please consult that document.
The rest of this document describes how to realize the Event Unsubscribe service using HTTP \cite{fielding2014hypertext}, TLS \cite{rescorla2018transport} and JSON \cite{bray2014json}, both in terms of its functions (Section \ref{sec:functions}) and its information model (Section \ref{sec:model}).

\newpage

\section{Service Functions}
\label{sec:functions}

This section lists the functions that must be exposed by the Event Unubscribe service in alphabetical order.
In particular, each subsection first names the HTTP method and path used to call the function, after which it names an abstract function from the Event Unsubscribe SD document, as well as input and output types.
All functions in this section respond with the HTTP status code \texttt{200 Created} if called successfully. The error codes are, \texttt{400 Bad Request} if request is malformed, \texttt{401 Unauthorized} if improper client side certificate is provided, \texttt{500 Internal Server Error} if Service Registry is unavailable.

\fsubsection{Query}{DELETE}{/eventhandler/unsubscribe\\\hspace*{0.1\textwidth}?address=\iparam{\{address\}}\\\hspace*{0.1\textwidth}\&port=\iparam{\{port\}}\\\hspace*{0.1\textwidth}\&event\_type=\iparam{\{event\_type\}}\\\hspace*{0.1\textwidth}\&system\_name=\iparam{\{system\_name\}}}{UnsubscribeRequest}{}

Called to unsubscribe from an event type, as exemplified in Listing \ref{lst:unsubscribe}.

\begin{lstlisting}[language=http,label={lst:subscribe},caption={An \fref{Event Subscribe} invocation.}]
DELETE /eventhandler/unsubscribe?address=192.168.0.1&event_type=EVENT_TYPE_1&port=9009&system_name=test_consumer HTTP/1.1
Accept: application/json

\end{lstlisting}
\newpage

\section{Information Model}
\label{sec:model}

Here, all data objects that can be part of the service calls associated with this service are listed in alphabetic order.
Note that each subsection, which describes one type of object, begins with the \textit{struct} keyword, which is meant to denote a JSON \pref{Object} that must contain certain fields, or names, with values conforming to explicitly named types.
As a complement to the primary types defined in this section, there is also a list of secondary types in Section \ref{sec:model:primitives}, which are used to represent things like hashes, identifiers and texts.

\msubsection{struct}{UnsubscribeRequest}

Identifies a requested Unsubscribe call.
As the fields of this type occur only as query parameters in the \fref{Unsubscribe} function, there is no need for it to representable as a JSON \pref{Object}.
This subsection exists only for the sake of completeness.

\subsection{Primitives}
\label{sec:model:primitives}

As all messages are encoded using the JSON format \cite{bray2014json}, the following primitive constructs, part of that standard, become available.
Note that the official standard is defined in terms of parsing rules, while this list only concerns syntactic information.
Furthermore, the \pref{Object} and \pref{Array} types are given optional generic type parameters, which are used in this document to signify when pair values or elements are expected to conform to certain types. 

\begin{table}[ht!]
\begin{tabularx}{\textwidth}{| p{3cm} | X |} \hline
\rowcolor{gray!33} JSON Type & Description \\ \hline
& \\ \hline
\end{tabularx}
\end{table}




\newpage

\bibliographystyle{IEEEtran}
\bibliography{bibliography}

\newpage

\section{Revision History}
\subsection{Amendments}

\noindent\begin{tabularx}{\textwidth}{| p{1cm} | p{3cm} | p{2cm} | X | p{4cm} |} \hline
\rowcolor{gray!33} No. & Date & Version & Subject of Amendments & Author \\ \hline

1 & 2020-12-05 & 1.0.0 & & Szvetlin Tanyi \\ \hline

\end{tabularx}

\subsection{Quality Assurance}

\noindent\begin{tabularx}{\textwidth}{| p{1cm} | p{3cm} | p{2cm} | X |} \hline
\rowcolor{gray!33} No. & Date & Version & Approved by \\ \hline

1 & & & \\ \hline

\end{tabularx}

\end{document}